%www.miktex.org
%\documentclass[a4paper,11pt]{article}
\documentclass[12pt,a4paper,titlepage]{article}%set letterpaper for letter format
%\documentclass[12pt]{report}
%\usepackage{wrapfig}
%\usepackage{shadow}
%\usepackage{epsf}
\usepackage{epsfig}
\usepackage{multicol}
\usepackage{color}
%\usepackage{fancyheadings}
\usepackage{fancyhdr}
\usepackage{fancybox}
%\usepackage{pstricks}
%\usepackage{textpath}
%\usepackage{pst-text}
\usepackage{amsmath}
\usepackage{amssymb}
\usepackage{theorem}
%\usepackage{french}ty

\usepackage[utf8]{inputenc}
\usepackage[french]{babel}
\usepackage[utf8]{inputenc}
\usepackage[french]{babel}
\usepackage[T1]{fontenc}
\usepackage{times}
\usepackage{pifont}
\usepackage{bbding}
\usepackage{wasysym}
\usepackage{manfnt}
\usepackage{hyperref}
\usepackage{glossaries}

% recopie partielle de modele.tex
\setcounter{secnumdepth}{3}
\newfont{\manfnt}{manfnt}

\sloppy
\flushbottom

\parindent 0.8cm
\leftmargini 2em
\leftmarginv .5em
\leftmarginvi .5em
\marginparwidth 20pt
\marginparsep 10pt
\oddsidemargin -0.77cm
\evensidemargin -0.77cm
\topmargin       0mm
\headheight      5mm
\voffset	-14mm
\headsep         4mm
\textheight 24.0cm
\textwidth  18.0cm

%\setcounter{secnumdepth}{3}
%\newfont{\manfnt}{manfnt}
%\setlength{\textwidth}{18.0cm}
%\setlength{\textheight}{24.0cm}
%\setlength{\oddsidemargin}{-0.8cm}
%\setlength{\parindent}{0.8cm}
%\setlength{\headsep}{0.4cm}
%\setlength{\headheight}{-1.25cm}

\newcommand{\ima}[2]{\epsfxsize=#1\epsfbox{#2}}
\makeatletter
\def\@normalsize{\@setsize\normalsize{12pt}\xpt\@xpt
\abovedisplayskip 10pt plus2pt minus5pt\belowdisplayskip \abovedisplayskip
\abovedisplayshortskip \z@ plus3pt\belowdisplayshortskip 6pt plus3pt
minus3pt\let\@listi\@listI}
\def\subsize{\@setsize\subsize{8pt}\xipt\@xipt}
\def\section{\@startsection {section}{1}{\z@}{12pt plus 2pt minus 2pt}
{12pt plus 2pt minus 2pt}{\bf}}
\def\subsection{\@startsection {subsection}{2}{\z@}{12pt plus 2pt minus 2pt}
{12pt plus 2pt minus 2pt}{\bf}}
\makeatother
\def\thesection       {\Roman{section}.}
\def\thesubsection    {\thesection\arabic{subsection}.}
\def\thesubsubsection {\thesubsection\arabic{subsubsection}.}
\newlength{\mylength}
\renewcommand{\baselinestretch}{1.1}
% fin recopie de modele.tex
\pagestyle{plain}

\setlength{\parindent}{0cm}

\title{\textbf{ Mise en place d'un cluster}}
\author{ Bardot Jérôme - Couturier Andy \\ Université de Moncton \\ Département d'informatique \\ Info 4025 }


\makeglossaries

%%%%%%%%%%%%%%%%%%%%%%%%%%%%%
%%%%% DÉBUT DU DOCUMENT %%%%%
%%%%%%%%%%%%%%%%%%%%%%%%%%%%%
\begin{document}

%\thispagestyle{empty}
%\include{page1}
%\newpage

\pagenumbering{roman}


\pagestyle{fancy}
\pagenumbering{arabic}

\fancyhead[l]{\small \emph{INFO4025}}
\fancyhead[c]{\small \emph{-- C --}}
%\fancyhead[r]{\small \emph{Laboratoire n$^{\circ}$ 2}}
\fancyhead[r]{\small \emph{- Présentation détaillée -}}

\fancyfoot[l]{\small \emph{}}
\fancyfoot[c]{\it \mancube \quad \thepage-\pageref{lastpage} \quad \mancube}
\fancyfoot[r]{\small \emph{}}

%%%%%%%%%%%%%%%%%%%%%%%%%%%%%%%%%%%%%%%%%%%%%%

\maketitle 

\tableofcontents
%\listoffigures
%\listoftables
\printglossaries

\section{\large Introduction}
\subsection{Les Machines Distribuées}
\subsection{Étude de cas concrets}
\section{\large Architecture et Organisation physique}
\section{\large Aspects Logiciel et Exploitation du parallélisme}
\section{\large Conclusion, tendance actuelle et future.}


\newpage

L'objectif du projet est de réaliser un \emph{CMS} \footnote{Content Management System} permettant à une communauté d'exister et de communiquer.



-  \href{http://book.cakephp.org/2.0/fr/index.html}{cakephp} \\
- \href{http://getbootstrap.com/}{bootstrap} \\


people don‚t know about
alternative \gls{computer} operating systems:
\glspl{Linux}, BSDs and GNU/Hurd.



\newglossaryentry{computer}
{
name=computer,
description={is a programmable machine that receives input,
stores and manipulates data, and provides
output in a useful format}
}

\newglossaryentry{Linux}
{
description={is a generic term referring to the family of Unix-like
computer operating systems that use the Linux kernel}
}

\printglossary


\label{lastpage}

\end{document}
